\documentclass[a4paper]{scrartcl}

\usepackage[utf8]{inputenc}
\usepackage[ngerman]{babel}
\usepackage{amsmath}
\usepackage{amssymb}
\usepackage{fancyhdr}
\usepackage{color}
\usepackage{graphicx}
\usepackage{lastpage}
\usepackage{listings}
\usepackage{tikz}
\usepackage{pdflscape}
\usepackage{subfigure}
\usepackage{float}
\usepackage{polynom}
\usepackage{hyperref}
\usepackage{tabularx}
\usepackage{forloop}
\usepackage{geometry}
\usepackage{listings}
\usepackage{fancybox}
\usepackage{tikz}
\usepackage{enumitem}
\usepackage{xcolor}

\input kvmacros

%Größe der Ränder setzen
\geometry{a4paper,left=3cm, right=3cm, top=3cm, bottom=3cm}

\fancyfoot[L]{}
\fancyfoot[C]{}
\fancyfoot[R]{Seite \thepage /\pageref*{LastPage}}

%Formatierung der Überschrift, hier nichts ändern
\def\header#1#2{
	\begin{center}
		{\Large Übungsblatt #1}\\
		{(Abgabetermin #2)}
	\end{center}
}

%Definition der Punktetabelle, hier nichts ändern
\newcounter{punktelistectr}
\newcounter{punkte}
\newcommand{\punkteliste}[2]{%
	\setcounter{punkte}{#2}%
	\addtocounter{punkte}{-#1}%
	\stepcounter{punkte}%<-- also punkte = m-n+1 = Anzahl Spalten[1]
	\begin{center}%
		\begin{tabularx}{\linewidth}[]{@{}*{\thepunkte}{>{\centering\arraybackslash} X|}@{}>{\centering\arraybackslash}X}
			\forloop{punktelistectr}{#1}{\value{punktelistectr} < #2 } %
			{%
				\thepunktelistectr &
			}
			#2 &  $\Sigma$ \\
			\hline
			\forloop{punktelistectr}{#1}{\value{punktelistectr} < #2 } %
			{%
				&
			} &\\
			\forloop{punktelistectr}{#1}{\value{punktelistectr} < #2 } %
			{%
				&
			} &\\
		\end{tabularx}
	\end{center}
}


\begin{document}
	
	{\LARGE \textbf{Beispielaufgabe: Minimierung}}
	
	\begin{figure}[H]
		\centering
		\begin{tabular}{c|*{4}{c}*{4}{|c}}
			\# & $x_3$ & $x_2$ & $x_1$ & $x_0$ & $f_1$ & $f_2$ & $f_3$ & $f_4$ \\
			\hline
			 0 & 0 & 0 & 0 & 0 & 1 & 1 & 1 & * \\
			 1 & 0 & 0 & 0 & 1 & 0 & 1 & 0 & 0 \\
			 2 & 0 & 0 & 1 & 0 & 0 & 0 & 1 & * \\
			 3 & 0 & 0 & 1 & 1 & 1 & 0 & 0 & 1 \\
			 4 & 0 & 1 & 0 & 0 & 0 & 1 & 1 & 1 \\
			 5 & 0 & 1 & 0 & 1 & 1 & 1 & 0 & 1 \\
			 6 & 0 & 1 & 1 & 0 & 1 & 0 & 1 & 0 \\
			 7 & 0 & 1 & 1 & 1 & 0 & 0 & 0 & 0 \\
			 8 & 1 & 0 & 0 & 0 & 0 & 1 & 1 & 1 \\
			 9 & 1 & 0 & 0 & 1 & 1 & 0 & 0 & 0 \\
			10 & 1 & 0 & 1 & 0 & 1 & 0 & 0 & 1 \\
			11 & 1 & 0 & 1 & 1 & 0 & 1 & 1 & 0 \\
			12 & 1 & 1 & 0 & 0 & 1 & 1 & 1 & * \\
			13 & 1 & 1 & 0 & 1 & 0 & 0 & 0 & 1 \\
			14 & 1 & 1 & 1 & 0 & 0 & 0 & 0 & 0 \\
			15 & 1 & 1 & 1 & 1 & 1 & 1 & 1 & *
		\end{tabular}
	\end{figure}
	
	\begin{enumerate}
		\item Geben Sie $f_1$, $f_2$, $f_3$ als DMF an (nicht mit dem Quine McCluskey-Verfahren).
		\item Geben Sie $f_4$ als DMF und KMF (nutzen sie KVs)
		\item Welche beiden Funktionen profitieren von einer gemeinsamen Implementierung? (also zusammen in einer Schaltung)  \\ 
		Wie nennt man dieses Vorgehen/diese Terme? \\
		Geben Sie die entsprechenden Terme an.
		\item Minimieren Sie die nicht in \textit{3.} verwendete Funktion mit dem Quine McCluskey-Verfahren.
		
	\end{enumerate}

	\newpage
	
	%Solution
	
	{\LARGE \textbf{Lösung}}
	
	\begin{enumerate}
		\item Minimierung durch KVs: 
		\begin{figure}[H]
			\centering
			\karnaughmap{4}{$f_1$}{{$x_3$}{$x_2$}{$x_1$}{$x_0$}}{1001011001101001}{%
				\put(1.5,0.5){\oval(0.9, 0.9)}
				\put(3.5,0.5){\oval(0.9, 0.9)}
				\put(0.5,1.5){\oval(0.9, 0.9)}
				\put(2.5,1.5){\oval(0.9, 0.9)}
				\put(1.5,2.5){\oval(0.9, 0.9)}
				\put(3.5,2.5){\oval(0.9, 0.9)}
				\put(0.5,3.5){\oval(0.9, 0.9)}
				\put(2.5,3.5){\oval(0.9, 0.9)}
			}
			\karnaughmap{4}{$f_2$}{{$x_3$}{$x_2$}{$x_1$}{$x_0$}}{1100110010011001}{%
				\put(0,0){\oval(1.9, 1.9)[rt]}
				\put(0,4){\oval(1.9, 1.9)[rb]}
				\put(4,0){\oval(1.9, 1.9)[lt]}
				\put(4,4){\oval(1.9, 1.9)[lb]}
				\put(2,3.5){\oval(3.9, 0.9)}
				\put(2,1.5){\oval(1.9, 0.9)}
			}
		\end{figure}
		
		\begin{figure}[H]
			\centering
			\karnaughmap{4}{$f_3$}{{$x_3$}{$x_2$}{$x_1$}{$x_0$}}{1010101010011001}{%
				\put(0,0){\oval(1.9, 1.9)[rt]}
				\put(0,4){\oval(1.9, 1.9)[rb]}
				\put(4,0){\oval(1.9, 1.9)[lt]}
				\put(4,4){\oval(1.9, 1.9)[lb]}
				\put(0,3){\oval(1.9, 1.9)[r]}
				\put(4,3){\oval(1.9, 1.9)[l]}
				\put(2,1.5){\oval(1.9, 0.9)}
			}
			\karnaughmap{4}{$f_4$}{{$x_3$}{$x_2$}{$x_1$}{$x_0$}}{*0*111001010*10*}{%
				\put(0.5,2){\oval(0.9, 3.9)}
				\put(1,2.5){\oval(1.9, 0.9)}
				\put(3,4){\oval(1.9, 1.9)[b]}
				\put(3,0){\oval(1.9, 1.9)[t]}
			}
		\end{figure}
	
		$\Rightarrow DMF_{f_1} =$ 
		$\overline{x_3}$ $\overline{x_2}$ $\overline{x_1}$ $\overline{x_0}$ $\vee$
		$\overline{x_3}$ $\overline{x_2}$ $x_1$ $x_0$ $\vee$
		$x_3$ $\overline{x_2}$ $x_1$ $\overline{x_0}$ $\vee$
		$x_3$ $\overline{x_2}$ $\overline{x_1}$ $x_0$ $\vee$
		$\overline{x_3}$ $x_2$ $x_1$ $\overline{x_0}$ $\vee$
		$\overline{x_3}$ $\overline{x_2}$ $x_1$ $x_0$ $\vee$
		$x_3$ $x_2$ $x_1$ $x_0$
		
		$\Rightarrow DMF_{f_2} =$ 
		$\overline{x_1}$ $\overline{x_0}$ $\vee$
		$\overline{x_3}$ $\overline{x_1}$ $\vee$
		$x_3$ $x_1$ $x_0$
	
		$\Rightarrow DMF_{f_3} =$ 
		$\overline{x_1}$ $\overline{x_0}$ $\vee$
		$\overline{x_3}$ $\overline{x_0}$ $\vee$
		$x_3$ $x_1$ $x_0$
		
		\item 
		$\Rightarrow DMF_{f_4} =$ 
		$\overline{x_2}$ $\overline{x_0}$ $\vee$
		$x_2$ $\overline{x_1}$ $\vee$
		$\overline{x_3}$ $\overline{x_2}$ $x_1$
		
		$\Rightarrow KMF_{f_4} =$ 
		$(\overline{x_2} \vee \overline{x_1}) \wedge 
		(\overline{x_3} \vee x_2 \vee \overline{x_0}) \wedge 
		(x_2 \vee x_1 \vee x_0)$
		
		
		\item $f_2$ und $f_3$ profitieren Bündelminimierung. Die sog. Koppelterme $\overline{x_1}$ $\overline{x_0}$ $\vee$
		 und $x_3$ $x_1$ $x_0$ müssen nur einmal realisiert werden. \\
		 
		 \item Minimiere $f_1$:
		 
		 \begin{minipage}{0.4\linewidth}
		 	\begin{figure}[H]
		 		\begin{tabular}{c|cccc|c}
		 			\# & $x_3$ & $x_2$ & $x_1$ & $x_0$ & M \\
		 			\hline
		 			\hline
		 			0 & 0 & 0 & 0 & 0 & \\
		 			\hline
		 			3 & 0 & 0 & 1 & 1 & \\
		 			5 & 0 & 1 & 0 & 1 & \\
		 			6 & 0 & 1 & 1 & 0 & \\
		 			9 & 1 & 0 & 0 & 1 & \\
		 			10 & 1 & 0 & 0 & 0 & \\
		 			12 & 1 & 1 & 0 & 0 & \\
		 			\hline
		 			15 & 1 & 1 & 1 & 1 & 
		 		\end{tabular}
		 	\end{figure}
		 \end{minipage}
		 \begin{minipage}{0.5\linewidth}
		 	$f_1$ besitzt nur Implikanten mit keiner, zwei oder vier Einsen. Somit gibt es keine Implikanten, die sich in nur einer Stelle unterscheiden. Es entstehen keine neuen Implikanten, wodurch man die zu \textit{1.} äquivalente DMF erhält: \\ 
		 	$\overline{x_3}$ $\overline{x_2}$ $\overline{x_1}$ $\overline{x_0}$ $\vee$
		 	$\overline{x_3}$ $\overline{x_2}$ $x_1$ $x_0$ $\vee$
		 	$x_3$ $\overline{x_2}$ $x_1$ $\overline{x_0}$ $\vee$
		 	$x_3$ $\overline{x_2}$ $\overline{x_1}$ $x_0$ $\vee$
		 	$\overline{x_3}$ $x_2$ $x_1$ $\overline{x_0}$ $\vee$
		 	$\overline{x_3}$ $\overline{x_2}$ $x_1$ $x_0$ $\vee$
		 	$x_3$ $x_2$ $x_1$ $x_0$
		 \end{minipage}
		 
	\end{enumerate}
	
	
	
\end{document}
%%% Local Variables:
%%% mode: latex
%%% TeX-master: t
%%% End:
